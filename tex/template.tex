\documentclass[11pt]{article}
\usepackage{xcolor}
\usepackage{fullpage}
\usepackage[colorlinks, allcolors=blue]{hyperref}
\usepackage{listings}
\usepackage{parskip}
\usepackage{indentfirst,graphicx}

\newcommand{\wei}[1] {#1 \texttt{wei}}

\date{}
%\title{Title of Proposal}
%\pagestyle{fancy}
%\lhead{}
%\rhead{\includegraphics[scale=0.2]{Nebulas.png}}

\lstdefinelanguage{diff}{
  morecomment=[f][\color{blue}]{@@},           % group identifier
  morecomment=[f][\color{red}]{-},             % deleted lines
  morecomment=[f][\color{green!50!black}]{+},  % added lines
  morecomment=[f][\color{magenta}]{---},       % diff header lines
  morecomment=[f][\color{magenta}]{+++},
}

\lstdefinestyle{base}{
  language=c++,
  breaklines=true,
  basicstyle=\ttfamily\color{black},
  moredelim=**[is][\color{green!50!black}]{@}{@},
}

\newcommand{\added}[1]{\color{green}{#1}}

\lstset{
  basicstyle=\footnotesize\ttfamily,
}


\begin{document}
\begin{titlepage}
    \begin{center}
        \vspace*{1cm}

\begin{flushright}
  \begin{tabular}{ll}
  \textbf{Document number}:& N1 \\
  \textbf{Date}:       & 2019-04-03 \\
  \textbf{Project}:    & Nebulas Mainnet \\
  \textbf{Reply-to}:   & {Your Name}\\
              &
              \textless\href{mailto:your@email.com}{your@email.com}\textgreater
  \end{tabular}
\end{flushright}

\vspace{2cm}
        \Huge
        \textbf{Title of Proposal}

        \vspace{0.5cm}

        \vfill
        \small \textbf{Note: you may put some note here!} \\
        \vspace{0.5cm}
        \includegraphics[scale=0.3]{Nebulas.png}
    \end{center}
\end{titlepage}

\section{Introduction}
A very brief high level view of your proposal, understandable by community members who are not necessarily experts in whatever domain you are addressing.


\section{Motivation and Scope}
Why is this important? What kinds of problems does it address? What is the intended user community? What level of programmers (novice, experienced, expert) is it intended to support? What existing practice is it based on? How widespread is its use? How long has it been in use? Is there a reference implementation and test suite available for inspection?



\section{Impact on the Nebulas Protocol}
What other components does does it depend on, and what depends on it? Is it a
pure extension, or does it require changes to Nebulas protocol? How does it
affect the communities, like Nebulas mainnet, exchanges, wallets, developers or
users?


\section{Design Choices}

Why did you choose the specific design that you did? What alternatives did you
consider, and what are the tradeoffs? What are the consequences of your choice,
for users and implementers? What decisions are left up to implementers? If
there are any similar designs, how do their design decisions compare to yours?

\section{Technical Specifications}

The community needs specifications to be able to fully evaluate your proposal.
For an initial proposal there are several possibilities:

\begin{itemize}
\item Provide some limited technical documentation. This might be OK for a very
  simple proposal such as a single function, but for anything beyond that the
    community will likely ask for more detail.

\item Provide technical documentation that is complete enough to fully evaluate
  your proposal. This documentation can be in the proposal itself or you can
    provide a link to documentation available on the web. If the community
    likes your proposal, they will ask for a revised proposal with formal
    wording.


\item Provide proof for the impact of your proposal. The community needs fully
  understand your proposal and the potential impact. This may be data
    simulation, experiment, code auditing result or users feedback.

\end{itemize}
\section{Acknowledgements}

\section{References}

\end{thebibliography}

\end{document}
